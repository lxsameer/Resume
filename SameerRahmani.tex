\documentclass[a4paper,11pt]{article}

\usepackage{hyperref}

\author{Sameer Rahmani}
\title{Curriculum vitae}
\begin{document}

\maketitle
\newpage
\section*{Sameer Rahmani}
\textbf{Address}: No353, Eastern Anoushirvan st, Mehrvila, Karaj, Alborz, Iran\\
\textbf{Phone Number}: +93358180918\\
\textbf{Email}: lxsameer@gnu.org\\
\textbf{Website}: lxsameer.com\\

\section*{Objective:}
I'm always looking forward to learn more and improve my knowledge and skills.
I like to challenge myself with using my knowledge to create new software or
invent new stuff which people shall use and remember me by. Obtaining a job
position which allow me to take pleasure from working is my main professional
life goal.


\section*{Experience:}
\begin{itemize}
\item \textbf{Senior web Developer and Project manager in Yellowen}\\
  \emph{2005 – Present}\\
  I'm the head developer and leader of several projects in Yellowen.
  I manage different teams with different talented developers in different
  operation fields. We create and launch web/mobile application, software automation and
  backend service project. My teams productivity performance improved by 60 percent
  by using Lean project management methodology. I'm writing a series of articles about
  my way of leadership and development in Yellowen.

  I'm responsible to design and develop high performance and scalable web application
  from scratch. I manage to create several reliable services by taking advantage of
  free and open source software which was a big success.

  Back in the early days of my work in Yellowen I was a member of web development team.
  We were working on school automation and management software of ministry of education
  which runs on about 3000 schools across the country.

  As result of a Yellowen strategy to use and develop free software there is a long list
  of free software we developed in Yellowen which I'm proud of.

\item \textbf{Web Developer in Bayan co}
  \emph{2009-2010}

A member of Python Web Development team and working on HOD email service and blog.ir project.
I was responsible for creating a reliable, scalable, distributed computing backend, elegant web
interface for HOD email service and also creating the beating heart of all the Bayan services
which was the central authentication and authorization service. I created a background job
 service to handle operation a big database using Django and Celery.

\item 4 years experience in teaching GNU/Linux and Programming in Aptech.
\item Network administrator chef at:
  \begin{itemize}
  \item Karajlug Community
  \item Notrica co
  \item Tehran Telecom co
  \item Karanic Group
  \item Hamshahri newspaper
  \end{itemize}
\end{itemize}

\section*{Education}
Islamic Azad University of Yazd\\
Bachelor's degree, Civil Engineering, \emph{2005 – 2009}\\

\section*{Languages}
English\\
Persian\\
Japanese (beginner)\\
Spanish (beginner)\\


\section*{Skills}
\begin{itemize}
\item Linux server and devices administration
\item Linux network administration
\item Distributed software development
\item Programing in:
  \begin{itemize}
  \item C/C++
  \item Python
  \item Ruby
  \item Perl
  \item PHP
  \item Lua
  \item Lisp
  \item Javascript
  \item Scala
  \item Clojure
  \end{itemize}

\item Good level of security knowledge
\item Linux kernel development
\item Over 13 years of programming experience
\item Web application development. Expert in popular frameworks like: Django, Ruby On Rails, Sinatra, Flask, Emberjs, Backbone, AngularJS etc
\item Expert in PostgreSQL, MySQL, Redis, MongoDB, Casandra database engines.
\end{itemize}


\section*{Activities}
\begin{itemize}
\item Proud member of GNU project.
\item Member of Debian GNU/Linux Python team.
\item Proud member of Karaj Linux users group. (Karajlug.org)
\item Co-founder of IranOnRails ruby users group. (First Iranian RUG)
\item One of the members of IrPycon 2013 organizers team.
\item One of the members of KFSC 2010, 2011, 2012, 2013 organizers team.
\item Many lectures and technical talks in local conferences
\end{itemize}

\section*{Projects}
I let my work to speak for itself. Here are some of my free projects
or those which I participated in:\\

\begin{itemize}

\item Faalis (\url{https://github.com/lxsameer/Faalis})\\
  A ruby on rails engine which provides a basic web application to
  use with other ruby on rails applications.

I'm one of the core developers and manager of Faalis. I designed and implemented most
of core functionality of Faalis like: Dashboard Interface, Dynamic Authorization service,
Awesome resource generators of Faalis and others systems.

\item Susanoo (\url{https://github.com/Yellowen/Susanoo})\\
  Meta Framework to develop hybrid mobile application in a blink using
  HTML5 and Javascript

  I started this project as a hobby but very soon Susanoo grew and become
  a full fledge mobile application development framework with lots of cool
  features. So we continue to work on Susanoo as team.

  Any RubyOnRails developer can get start and create awesome mobile apps
  in a blink of an eye using Susanoo because it is just like ROR.

\item Djamo (\url{https://github.com/Yellowen/Djamo})\\
  An ORM-like object mapper for Django and MongoDB which aimed for
  simplicity and easy usage.

  After wandering around to find a suitable DRM for we decided to create our
  own and we give birth to Djamo which is faster and better organized compared
  to their rivals. I designed and developed Djamo entirely.

\item Daarmaan (\url{https://github.com/Yellowen/daarmaan})\\
  A free Single Sign On software. A complete solution to implement
  central authentication in a cloud.

  There are some well tuned robust central authentication solution
  out there, but none of them fits our needs. My mission was to create
  a central authentication service with a single sign on interface.
  Daarmaan is the result. A single sign on solution which fits our
  requirements and work perfectly.

\item Model discovery (\url{https://github.com/lxsameer/model_discovery})\\
  A solution for Ruby on Rails applications model discovery problem.
  It's simple yet very handy.

  Because of RubyOnRails autoload mechanism It's impossible to get a list
  of your web application modules without walking through the filesystem
  and read the file name which can be very slow and also you can't discover
  models in other gems.

  Model Discovery create a list of all models in a web application whether it
  defined in web app directly or inside a gem and store them in a database so
  developers can access them easily.

  I designed Model Discovery to use with the Faalis authorization system. But
  now it has lots of users for its own.

\item Site framework (\url{https://github.com/lxsameer/site_framework})\\
  A site framework for Ruby on Rails web framework inspired by
  Django site fremework.

  Ruby on Rails is missing a site framework to create web applications which handle
  multiple web sites at once. We needed to implement such functionality for a startup
  so I implemented Site Framework by looking back at Django.

  I wrote some Rack middleware and an API to load views and models base on the current
  requested website.

\item Vakhshour (\url{https://github.com/Yellowen/Vakhshour})\\
  Very simple and handy network event passing application

  I was looking for a secure and lightweight message passing service
  to implement a couple of the core functionalities of Daarmaan. But
  I didn't find anything to satisfy my needs. So I designed a very simple
  yet secure framework which use two way SSL connections along side popular
  message passing solutions like event subscription and pushing.

\item RedVakhshour (\url{https://github.com/Yellowen/red-vakhshour})\\
  Ruby implementation of Vakhsour client.

  To implement a Vakhshour client in Ruby I needed to implement AMP
  protocol in Ruby ( Ramp was the result ) Then i created the RedVakhshour
  to work with Rails applications.

\item Kuso IDE (\url{https://github.com/KusoIDE/KusoIDE})\\
  A Gnu/Emacs base IDE with special goals

  I love Emacs. It's the best editor around for me. I created an
  Emacs distribution with certain goals in mind like: portability.
  Kuso has a solid core and easy to extend package manager.

\item MBTilesViewer (\url{https://github.com/lxsameer/MBTilesViewer})\\
  View geodata offline in MBTiles format.

  Our GIS customers did not have access to fast and stable internet
  connection everywhere and they need to access their GIS data everywhere
  like caves and etc.

  MbTile is the universal standard to store GIS data within SQLite databases.
  Unfortunately, there is no viewer for MbTiles around. So I create one using Nodejs,
  webkite and openlayer.

\end{itemize}



\section*{Interests}
\begin{itemize}

\item KyoKushin karate\\
  I’m a professional fighter and I enjoy training everyday. I have some national
  level titles. It help me to keep my body strong and my mind healthy.

\item Watching Japanese anime
\item Drawing comics
\item Playing guitar.
\item Creating new hardwares
\item Playing computer games
\end{itemize}

\section*{References}
\textbf{My website}: \url{http://www.lxsameer.com/}\\
\textbf{My Github profile}:  \url{https://github.com/lxsameer} (\textbf{Checkout organizations which I controll})\\
\textbf{Yellowen Github profile}: \url{https://github.com/Yellowen}\\
\textbf{Ohloh profile}: \url{https://www.ohloh.net/accounts/lxsameer}\\
\textbf{linkedin profile}: \url{http://www.linkedin.com/pub/sameer-rahmani/54/407/423}\\
\textbf{Careers}: \url{http://careers.stackoverflow.com/lxsameer}\\
\end{document}
